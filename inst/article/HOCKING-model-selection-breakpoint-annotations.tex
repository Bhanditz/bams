\documentclass{bioinfo}
\usepackage{xcolor}
\usepackage{multirow}
\usepackage{textcomp}
%slashbox for tables with \ in upper left for row and column dimnames
%\usepackage{slashbox}
\usepackage{booktabs}
%\usepackage{hyperref}
\copyrightyear{2012}
\pubyear{2012}
\newcommand{\url}[1]{\texttt{#1}}
\newcommand{\argmin}{\operatorname*{arg\, min}}
\newcommand{\model}[1]{#1}
\newcommand{\code}[1]{\texttt{#1}}
\newcommand{\package}[1]{\texttt{#1}}
\definecolor{cghseg.k.color}{HTML}{F8766D}
\definecolor{flsa.norm.color}{HTML}{619CFF}
\definecolor{flsa.color}{HTML}{00BA38}
\definecolor{dnacopy.alpha.color}{HTML}{D39200}
\definecolor{dnacopy.prune.color}{HTML}{00C19F}
\definecolor{dnacopy.sd.color}{HTML}{DB72FB}
\definecolor{glad.haarseg.color}{HTML}{93AA00}
\definecolor{glad.lambdabreak.color}{HTML}{00B9E3}
\definecolor{glad.MinBkpWeight.color}{HTML}{FF61C3}
\definecolor{glad.default.color}{HTML}{FFFFFF}
\definecolor{dnacopy.default.color}{HTML}{FFFFFF}
\definecolor{cghseg.mBIC.color}{HTML}{FFFFFF}
\definecolor{cghFLasso.color}{HTML}{FFFFFF}


% CATEGORY: Gene expression --- Original Paper

% sections Original papers: Title page, Structured Abstract,
% Introduction, System and methods, Algorithm, Implementation,
% Discussion, References.

% survey of recent papers: 

% introduction methods results discussion acknowledgments references
% appendix

% introduction approach-overview methods results discussion

% intro methods results discussion-and-conclusion ack refs

% intro algo methods results discussion

\newcommand{\NA}{\texttt{NA}}
\newcommand{\RR}{\mathbb R}
\newcommand{\NN}{\mathbb N}
 
\begin{document}
\firstpage{1}

\title[
Learning
smoothing models
with
breakpoint annotations
]{
Learning smoothing models of copy number profiles
  using breakpoint annotations}

\author[Hocking \textit{et~al}]{Toby Dylan
  Hocking\,$^{1,2,3,4}$\footnote{to whom correspondence should be
    addressed}, 
Gudrun Schleiermacher\,$^5$, 
Isabelle Janoueix-Lerosey\,$^{5}$, 
Olivier Delattre\,$^5$, Francis Bach\,$^1$ and
  Jean-Philippe Vert$^{2,3,4}$ }

\address{$^{1}$INRIA -- Sierra project-team, Paris, F-75013, 
$^{2}$ Centre for computational biology, 
Mines ParisTech, Fontainebleau, F-77300, 
$^{3}$ Institut Curie, 
$^{4}$ INSERM U900 and
$^{5}$ INSERM U830, Paris, F-75248, France}

\history{Received on XXXXX; revised on XXXXX; accepted on XXXXX}

\editor{Associate Editor: XXXXXXX}

\maketitle

\begin{abstract}
  % Abstracts are structured with a standard layout such that the text
  % is divided into sub-sections under the following five headings:
  % Motivation, Results, Availability and Implementation, Contact and
  % Supplementary Information. In cases where authors feel the
  % headings inappropriate, some flexibility is allowed. The abstracts
  % should be succinct and contain only material relevant to the
  % headings. A maximum of 150 words is recommended. If internet
  % hyperlinks are available for any part of the abstract, then this
  % should be given in the form of 'clickable text',
  % i.e.{{http://www...}}.
\section{Motivation:}
% Motivation: This section should specifically state the scientific
% question within the context of the field of study.
Many models have been proposed to detect breakpoints in chromosomal
copy number profiles, but it is usually not obvious to decide which is
most effective for a given data set. Furthermore, most methods have a
smoothing parameter that determines the number of breakpoints and must
be chosen using various heuristics.

\section{Results:}
We present three contributions toward automatic training of smoothing
models. First, we propose to select the model and degree of smoothness
that maximizes agreement with visual breakpoint region
annotations. Second, we develop cross-validation procedures to
estimate the error of the trained models. Third, we apply these
methods to a new database of annotated neuroblastoma copy number
profiles, which we make available as a public benchmark for testing
new algorithms. Whereas previous studies have been qualitative or
limited to simulated data, our approach is quantitative and suggests
which algorithms are fastest and most accurate in practice on real
data.

\section{Availability:}
% This section should state software availability if the paper focuses
% mainly on software development or on the implementation of an
% algorithm. Examples are: 'Freely available on the web at
% http://www.example_url.org.' Website implemented in Perl, MySQL and
% Apache, with all major browsers supported'; or 'Source code and
% binaries freely available for download at URL, implemented in C++
% and supported on linux and MS Windows'. The complete address (URL)
% should be given. If the manuscript describes new software tools or
% the implementation of novel algorithms the software must be freely
% available to non-commercial users. Authors must also ensure that the
% software is available for a full TWO YEARS following
% publication. The editors of Bioinformatics encourage authors to make
% their source code available and, if possible, to provide access
% through an open source license (see www.opensource.org for examples.
Copy number profiles can be annotated using a GUI:\\
\verb+http://pypi.python.org/pypi/annotate_regions+\\
The annotated neuroblastoma copy number profiles
are available in R: \code{data(neuroblastoma,package="neuroblastoma")}\\
\verb+http://cran.r-project.org/web/packages/neuroblastoma+

\section{Contact:}
% Full email address to be given, preferably an institution email
% address.
 \href{mailto:toby.hocking@inria.fr}{toby.hocking@inria.fr}
\end{abstract}

\section{Introduction: the need for smoothing model selection
  criteria}

DNA copy number alterations (CNAs) can result from various types of
genomic rearrangements, and are important in the study of many types
of cancer \citep{weinberg}. In particular, clinical outcome of
patients with neuroblastoma has been shown to be worse for tumors with
segmental alterations or breakpoints in specific genomic regions
\citep{isabelle-2009,gudrun-jclinicaloncology}.  Thus, to construct an
accurate predictive model of clinical outcome for these tumors, we
must first accurately detect the precise location of each breakpoint.


In recent years, array comparative genomic hybridization (aCGH)
microarrays have been developed as genome-wide assays for CNAs, using
the fact that microarray logratio is proportional to DNA copy number
\citep{pinkel}. In parallel, there have been many new mathematical
models proposed to smooth the noisy signals from these microarray
assays in order to recover the CNAs
\citep{glad,statistical-approach,dnacopy,cghFLasso,haarseg,fused-lasso-path}.
Each model has different assumptions about the data, and it is not
obvious to decide which model is appropriate for a given data set.

Furthermore, most models have parameters that control the degree
of smoothness. Varying these smoothing parameters will vary the number
of detected breakpoints. Most authors give default values that
accurately detect breakpoints on some data, but do not necessarily
generalize well to other data. There are some specific criteria for
choosing the degree of smoothness in some models
\citep{lavielle2005,mBIC,penalized-cna}, but the mathematical
assumptions of these models are not often verified in real noisy
microarray data, which can lead to poor estimation of CNAs.

In practice, software tools such as VAMP \citep{vamp} are often used to
normalize the noisy microarray signals, then plot them
% with some arbitrary smoothing model
 against genomic position for interpretation
by an expert biologist looking for CNAs. Indeed, to motivate the use
of their cghFLasso smoothing model, \cite{cghFLasso} write ``The
results of a CGH experiment are often interpreted by a biologist, but
this is time consuming and not necessarily very accurate.''

In contrast, the first contribution of this paper is the idea that the
expert interpretation of the biologist is very accurate and in fact
valuable for model selection. To tune the smoothness parameter in
practice, the biologist will often examine plots of the microarray
signal with a smoothed model, changing the smoothness parameter until
the model seems to capture all the visible breakpoints the data. In
Section~\ref{training}, we make this intuition concrete by defining a
training protocol based on visual annotations that can
quantify the accuracy of a smoothing method. We note that using
databases of visual annotations is not a new idea, and has been used
successfully for object recognition and detection in the field of
computer vision \citep{labelme}. In bioinformatics, \cite{shah}
proposed a hidden Markov model with position-specific prior
probabilities for copy number variations, but no previous models have
attempted to learn the degree of smoothness using breakpoint
annotations.

Our second contribution is a protocol to estimate the breakpoint
detection ability of the trained smoothing models on real data. In
Section~\ref{crossvalidation}, we propose to estimate the false
positive and false negative rates of the trained models using
cross-validation. This provides an objective criterion for deciding
which smoothing algorithms are appropriate for which data.

The third contribution of this paper is a systematic, quantitative
comparison of the accuracy of several common smoothing algorithms on a
new database of 575 annotated neuroblastoma copy number profiles,
which we give in Section~\ref{results}. For simulated data,
\cite{compare} compared GLAD, DNAcopy, and a hidden Markov model by
examining false positive and false negative rates for detection of a
breakpoint at a specific location. \cite{cnvfinder} used quantitative
PCR to validate the CNA-detection accuracy of their \model{CNVfinder}
method on real data. But quantitative PCR is low-throughput and
costly, and in real copy number profiles the locations of breakpoints
are unknown. So in fact there are no previous studies that
quantitatively compare breakpoint detection of smoothing models on
real data. In this paper we propose to use annotated regions instead
of precise breakpoint locations for quantifying smoothing model
accuracy, and we make available 575 new annotated neuroblastoma copy
number profiles as a benchmark for the community to test new
algorithms on real data.

Several authors have recently proposed methods for so-called joint
segmentation of multiple CGH profiles, under the hypothesis that each
profile shares breakpoints in the exact same location
\citep{jp-nips,nbc}. These models are not useful in our
setting, since we assume that breakpoints do not occur in the exact
same locations across copy number profiles. Instead, we focus on the
case of learning a model that will accurately detect an unknown number
of breakpoints in a copy number profile.

%\newpage
\begin{methods}
\section{Training smoothing models using breakpoint annotations}
\label{training}
The first contribution of this paper is a smoothing model training
protocol that uses visually determined breakpoint annotations to
quantify model accuracy, which we explain in this section.

\subsection{Detecting breakpoints using smoothing models}

Assume that we have observed $n$ chromosomal copy number profiles,
each with an unknown number of breakpoints at unknown locations. We
would like to recover the breakpoints accurately using a model with
parameter $\lambda$ that controls the degree of smoothness. As shown
in Figure~\ref{figure-smoothing}, we represent the $d$ probes on
chromosome $c$ of profile $i$ using the following numbers:
\begin{equation*}
  \label{eq:def}
  \begin{array}{ccccccl}
    p_1 & \leq & \dots & \leq & p_{d} &\in\NN&
    \text{positions on chromosome $c$}\\
    y_1 & , & \dots & , & y_{d} & \in\RR&
    \text{logratio measurements}\\
    \hat y_1^\lambda & , & \dots & , & \hat y_{d}^\lambda &\in\RR&
    \text{smoothed profile}
  \end{array}
\end{equation*}


% $(p_j,c_j)\in\NN\times\mathcal C$, where
%is the set of human
% chromosomes. The copy number profiles come from a variety of different
% microarray technologies, thus when profile $i$ does not have a probe
% that maps to genomic position $j$, we write $y_{ij}=\NA$.

We define the breakpoints predicted on this chromosome as the set of
positions where there are jumps in the smoothed signal:
\begin{equation}
  \label{eq:predicted}
  \hat B^\lambda_{ic} = \big\{
(p_j+p_{j+1})/2
\mid
\hat y_{j}^\lambda\neq
\hat y_{j+1}^\lambda,
\ \forall j=1,\dots,d-1
\big\}
\end{equation}

Then, we define $\hat B_i^\lambda$ to be the complete set of genomic
breakpoints predicted by algorithm $\lambda$ for profile $i$, over all
chromosomes $c$.



\subsection{Breakpoint annotations quantify model accuracy}

Intuitively, by visual inspection of the noisy signal, it is not
obvious to locate the exact location of a breakpoint, but it should be
easy to determine whether or not a region contains a breakpoint.
So rather than defining annotations in terms of precise breakpoint
locations, we instead define them in terms of regions. We define a
genomic region $R_k$ as
% \begin{equation*}\label{eq:region}
% R_k=\{(p,c)\in\NN\times\mathcal
% C\mid c=c_k\text{ and } \underline r_k\leq p\leq \overline r_k\},
% \end{equation*}
the subset of genomic positions on chromosome $c_k$ between the min
$\underline r_k$ and max $\overline r_k$. 

\begin{figure*}
\includegraphics[width=\textwidth]{figure-smoothing}
\caption{\label{figure-smoothing}Model agreement to annotated
   regions can be measured by examining the positions of
  predicted breakpoints $\hat B_{i}^\lambda$ (dashed vertical blue
  lines) observed in the smoothing model $\hat y^\lambda$ (solid
  blue lines). 
  Black circles show 
  logratio measurements $y$ plotted against
  position $p$ 
  for a single profile
  $i=375$. Chromosomes are shown in panels from
  left to right, and different values of the smoothing parameter
  $\lambda$ in the \model{flsa} model are shown in panels from top to
  bottom. Models with too many breakpoints ($\lambda=0.5$) and too few
  breakpoints ($\lambda=10$) are suboptimal, so we pick an
  intermediate model ($\lambda=7.5$) that maximizes agreement with
  the annotations, thus detecting a new breakpoint on chromosome 7
  which was not annotated.}
\end{figure*}

So, we define the breakpoint annotation for profile $i$ in region
$k$ as
\begin{equation}
  \label{eq:bik}
  b_{ik}=
  \begin{cases}
    0 & \text{if profile $i$ has no breakpoints in $R_k$}\\
    1 & \text{if profile $i$ has at least 1 breakpoint in $R_k$},
  \end{cases}
\end{equation}
which can be determined by visual inspection of the scatterplot of
logratio measurements $y$ versus position $p$, as in
Figure~\ref{figure-smoothing}.

The idea for model selection is to choose $\lambda$ such that the
predicted breakpoints $\hat B_i^\lambda$ agree with the annotations
$b_{ik}$, as shown in Figure~\ref{figure-smoothing}. To quantify this,
for each region $k$, we predict 0 if there are no predicted
breakpoints in the region, and 1 if there is at least 1 predicted
breakpoint:
\begin{equation}
  \label{eq:bikhat}
  \hat b_{ik}^\lambda =
  \begin{cases}
    0 & \text{ if } R_k\cap\hat B_i^\lambda = \emptyset\\
    1 & \text{ otherwise}
  \end{cases}
\end{equation}

We can measure the error of a model
at region $k$ on profile $i$ with the indicator function
\begin{equation}
  \label{eq:1error}
  E^k_i(\lambda)= 
  \begin{cases}
    0 & \text{ if }b_{ik}=\hat b_{ik}^\lambda\\
    1 & \text{ otherwise}
  \end{cases}
\end{equation}
and with respect to an entire profile $i$ using
\begin{equation}
  \label{eq:localerror}
  E_i^{\text{local}}(\lambda)=\sum_{k}E^k_i(\lambda)
\end{equation}
We define the local model as the model obtained by choosing a
different $\lambda_i$ to minimize Equation~\ref{eq:localerror} for
each profile $i$.
% These local models are presumably useful for detecting precise
% locations of breakpoints for profiles where we have some annotated
% regions, but are not useful for detecting breakpoints in un-annotated
% profiles.

However, we can learn a globally optimal smoothing parameter $\lambda$
by minimizing the error with respect to all the profiles:
\begin{equation}
  \label{eq:globalerror}
  E^{\text{global}}(\lambda)=\sum_{i=1}^n E_i^{\text{local}}(\lambda).
\end{equation}
We define the global model as the model obtained by choosing a
smoothing parameter $\lambda^*$ that minimizes
Equation~\ref{eq:globalerror}.

\begin{figure*}
\includegraphics[width=\textwidth]{figure-learning-curves}
\caption{\label{figure-learning-curves} Training error functions for
  global and local models plotted against smoothing parameter
  $\lambda$. In the top row panels, we plot
  $E^{\text{global}}(\lambda)$ from Equation~\ref{eq:globalerror}, and
  in the other rows, we plot $E^{\text{local}}_i(\lambda)$ from
  Equation \ref{eq:localerror}. Each column of plots shows the error
  of a particular algorithm, and the minimum chosen using the global
  training procedure is shown using a vertical grey line. Note that
  the local model training error can be reduced by moving from the
  globally optimal smoothing parameter $\lambda^*$ to a local value
  $\lambda_i$, as in profile $i=375$ for algorithms
  dnacopy.sd and glad.lambdabreak. For the local models trained on
  single profiles, there are at most 6 training examples, so many
  smoothing parameters attain the minimum. Thus, we use the protocol
  described in section \ref{pick} to pick the best value, shown as a
  black dot.}
\end{figure*}

\label{quantify}

\subsection{Picking the optimal degree of smoothness}

We assume that $\lambda$ is a tuning parameter that is monotonic in
the number of breakpoints, which is the case for the models considered
in this paper.

Fix a set of smoothing parameters, and run the
smoothing algorithm with each of these parameters. Intuitively, we
should pick the value of $\lambda$ that maximizes agreement with
annotation data. For global models, we attempt to minimize
Equation~\ref{eq:globalerror}, and there is usually one best value,
$\lambda^*$.

However, for the local model for profile $i$, we want to minimize the
local error as defined in Equation~\ref{eq:localerror}. Since the
training set consists of only the annotations of one profile $i$,
there may be no unique smoothing parameter $\lambda$ that minimizes
the error.
% \begin{equation}
%   \label{eq:estar}
% E^*=\min_{i\in\{1,\dots,\ell\}}E(\lambda_i).
% \end{equation}
% Let the set of parameters that minimize this error be
% $\lambda^*_1<\dots<\lambda^*_{\ell^*}$
\label{pick} We propose to pick between models that achieve the
minimum number of errors based on the shape of the error curve,
and these cases are illustrated in Figure
\ref{figure-learning-curves}.
% \begin{equation}
%   \label{eq:pick}
%   L(E)=
%   \begin{cases}
%     \lambda^* & \text{ if }E(\lambda_1)=E(\lambda_\ell)=E^*\\
%     \lambda^*_{\ell^*}&\text{ if }E(\lambda_1)=E^*\\
%     \lambda^*_{1} &\text{ if }E(\lambda_\ell)=E^*\\
%     \lambda^*_{\lceil \ell^*/2\rceil} & \text{ otherwise}
%   \end{cases}
% \end{equation}
\begin{enumerate}
\item When the minimum error is achieved in a range of intermediate
  parameter values, we pick a value in the middle. This occurs in the
  local error curves shown for \model{flsa.norm} and
  \model{cghseg.k}.
\item When the minimum is attained by the model with the most
  breakpoints, we pick the model with the fewest breakpoints that has
  the same error. This attempts to minimize the false positive
  rate. This occurs for profile $i=375$ with models
  \model{dnacopy.sd} and \model{glad.lambdabreak}.
\item When the minimum is attained by the model with the fewest
  breakpoints, we pick the model with the most breakpoints that has
  the same error. This attempts to minimize the false negative rate,
  and occurs for profile $i=362$ with models
  \model{dnacopy.sd} and \model{glad.lambdabreak}.
\end{enumerate}



\section{Estimating the error of the trained model using
  cross-validation}\label{crossvalidation}

% Let us turn to the question of whether local or global models are
% preferable. A priori it is not obvious whether the local or global
% procedures for model training should result in models that are more
% accurate breakpoint detectors. Intuitively, the global model is
% attractice since it allows sharing information between
% profiles. However, choosing the locally optimal parameter value
% instead of the globally optimal value will never increase the training
% error.

% But in fact we are really concerned with the test error or ability of
% the model to detect breakpoints in un-annotated regions.
The second contribution of this paper are cross-validation procedures
that estimate the generalization error of the trained smoothing
models. The main idea is to use a training set of annotations to learn
the parameter of a smoothing model, then quantify the error of the
model using a test set of annotations. Notably, this enables
quantitative comparison of breakpoint detection models on copy number
profiles from real microarray data.

\begin{figure*}[b]
\includegraphics[width=\textwidth]{figure-roc}
\caption{\label{figure-roc}ROC curves for the training error with
  respect to the breakpoint annotation data are shown as colored
  lines. The curves are shown in 3 panels zoomed to the upper left
  region of ROC space to avoid visual clutter. Each curve is traced
  by plotting the error of a model as the degree of smoothness is
  varied, and an empty black circle shows the global model chosen by
  minimizing the error with respect to all annotations. Algorithms
  with no tuning parameters are shown as black dots. Note that some
  ROC curves appear incomplete since some segmentation algorithms are
  not flexible enough for the task of breakpoint detection, even
  though we ran each algorithm on a very large range of smoothness
  parameter values.}
\end{figure*}


\begin{table}[b]
\begin{center}
\input{table-annotation-chrom-counts}
\end{center}
\caption{\label{table-annotation-chrom-counts} 
  Counts of normal and breakpoint annotations 
  in the neuroblastoma data set, 
  conditional on region. Min and max limits of each region 
  are shown in mega base pairs,
  in reference to the Hg19 Human genome assembly.}
\end{table}
%TODO http://www.oxfordjournals.org/our_journals/bioinformatics/for_authors/general.html
% In particular, microarray data should be submitted to one of the recognized public repositories in a MIAME compliant way (see C. A. Ball et al., Submission of Microarray Data to Public Repositories. PLoS Biology, 2, e317).

\subsection{Leave-one-out cross-validation for comparing local and
  global models}
To compare the breakpoint detection performance of local and global
models on un-annotated regions, we propose leave-one-out
cross-validation on regions.
\begin{itemize}
\item For each annotated region $k$:
\begin{enumerate}
\item Designate $R_k$ as the test region, and set aside the
  annotations in this region from all the profiles.
\item Using all the other annotations as a training set, pick the best
  $\lambda$ using the protocol described in Section~\ref{pick}. For
  local models we learn a profile-specific $\lambda_i$ that minimizes
  $E_i^{\text{local}}$, and for global models we learn a global
  $\lambda$ that minimizes $E^{\text{global}}$.
\item To estimate how the model generalizes, count the errors of the
  learned model in the test region $R_k$.
\end{enumerate}
\item To estimate the ability of the trained model to predict
  breakpoints at a general un-annotated region, take the mean test
  error over all regions. 
\end{itemize}

\subsection{$n/t$-fold cross-validation to estimate error on
  un-annotated profiles}

\begin{table}[hb!]
\begin{center}
\input{table-annotation-profile-counts}
\end{center}
\caption{\label{table-annotation-profile-counts}
  Counts of profiles in the neuroblastoma data set, 
  conditional on number of annotations.
  Note that most profiles have more normal regions than breakpoint regions. 
  For example, 335 profiles have all 6 regions annotated as normal.}
\end{table}



Since the annotation process is time-consuming, we are interested in
training an accurate breakpoint detector with as few annotations as
possible. Thus we would like to answer the following question: how
many profiles $t$ do I need to annotate before I get a global model
that will generalize well to all the other profiles?

To answer this question, we estimate the error of a global model
trained on the annotations from $t$ profiles using
cross-validation. We divide the set of $n$ annotated profiles into
exactly $\lfloor n/t \rfloor$ folds, each with approximately $t$
profiles. For each fold, we consider its annotations a training set
for a global model, and combine the other folds as a test set to
quantify the model error. The final estimate of generalization error
is then the average model error over all folds.



\section{Data and models}

\subsection{Neuroblastoma copy number data}\label{data}

We analyzed a new data set of $n=575$ copy number profiles from aCGH
microarray experiments on neuroblastoma tumors taken from patients at
diagnosis. The microarrays were produced using various technologies,
so do not all have the same probes. The number of probes per
microarray varies from 1719 to 71340. In this article we analyzed the
normalized logratio measurements of these microarrays, which we have
made available in R package \package{neuroblastoma} on CRAN.




Six chromosome arms known to be associated with prognostic impact were
annotated in the microarray data set \citep{isabelle-2009}. Each
region $R_1,\dots,R_6$ was defined by the start and end of a
chromosome arm, and the genomic coordinates of these regions are given
in Table~\ref{table-annotation-chrom-counts}.

For each profile $i$, our domain expert annotated each region $k$ by
examining the plotted profile in VAMP \citep{vamp} and recording 0 or
1 in a spreadsheet, according to the definition of breakpoint
annotations in Equation~\ref{eq:bik}.
Table~\ref{table-annotation-chrom-counts} shows counts of annotations
per region, and Table~\ref{table-annotation-profile-counts} shows
counts of annotations per profile. Some profiles have less than 6
annotations since we excluded regions where presence of breakpoints
could not be determined by visual inspection. The annotations are
shown as colored rectangles in Figure~\ref{figure-smoothing}, and are
available in R package \package{neuroblastoma} on CRAN.





\subsection{Smoothing models}\label{models}

In this study we considered smoothing models from the bioinformatics
literature with free software implementations available as R packages
on CRAN, R-Forge, or Bioconductor \citep{R,R-Forge,Bioconductor}.



We used version 1.03 of the \package{flsa} package from CRAN to
calculate the Fused Lasso Signal Approximator as described by
\cite{fused-lasso-path}. The FLSA solves the following optimization
problem for each chromosome:
\begin{equation}
  \label{eq:flsa}
  \hat y^\lambda = 
\argmin_{\beta\in\RR^d} 
\frac 1 2 \sum_{i=1}^d (y_i-\beta_i)^2
+\lambda_1\sum_{i=1}^d|\beta_i|
+\lambda_2\sum_{i=1}^{d-1}|\beta_i-\beta_{i+1}|.
\end{equation}
We define a grid of values $\lambda\in\{10^{-5},\dots,10^{12}\}$, take
$\lambda_1=0$, and consider the following parameterizations for
$\lambda_2$:
\begin{itemize}
\item \model{flsa}: $\lambda_2=\lambda$.
\item \model{flsa.norm}: $\lambda_2=\lambda d \times 10^6/l$ where
  $d$ is the number of points and $l$ is the length of the chromosome
  in base pairs.
\end{itemize}


We used version 1.29.0 of the \package{DNAcopy} package from
Bioconductor to fit the circular binary segmentation model of
\cite{dnacopy}. We varied the degree of smoothness by adjusting the
\code{undo.SD}, \code{undo.prune}, and \code{alpha} parameters of the
\code{segment} function.

We used version 0.2-1 of the \package{cghFLasso} package from CRAN,
which implements the method of \cite{cghFLasso}, but does not provide
any smoothness parameters for breakpoint detection.

We used version 2.17.0 of the \package{GLAD} package from Bioconductor
to fit the GLAD adaptive weights smoothing model of \cite{glad}. We
varied the degree of smoothness by adjusting the \code{lambdabreak}
and \code{MinBkpWeight} parameters of the \code{daglad} function. For
the \model{glad.haarseg} model, we used the
\code{smoothfunc="haarseg"} option and varied the \texttt{breaksFdrQ}
parameter to fit the wavelet smoothing model of \cite{haarseg}.

We used version 0.01 of the \package{cghseg} package from R-Forge to
fit the maximum-likelihood piecewise constant smoothing model of
\cite{statistical-approach} for each chromosome using pruned dynamic
programming \citep{pruned-dp}. We used the \code{segmeanCO} function
with \code{kmax=20} to obtain the maximum-likelihood piecewise
constant smoothing model $\hat y^k$ for $k= 1, \dots, 20$ segments.
\cite{lavielle2005} suggested penalizing $k$ breakpoints in a signal
sampled at $d$ points using $\lambda k$, and varying $\lambda$ as a
tuning parameter. We implemented this model selection criterion as the
\model{cghseg.k} model, for which we define the optimal number of
segments
\begin{equation}
  \label{eq:cghseg.k}
  k^*(\lambda) = 
\argmin_{k\in\{1,\dots,20\}}
\lambda k+  \frac 1 d \sum_{i=1}^d (y_i-\hat y_i^k)^2,
\end{equation}
and the optimal smoothing $
  \hat y^\lambda = \hat y^{k^*(\lambda)}.
$
%  For \texttt{cghseg.min.abs.diff}, we define a
% breakpoint to be significant if its magnitude is greater than some
% value $\lambda$, and we picked the smallest $k$ for which all
% breakpoints were significant. 
For the \model{cghseg.mBIC} model, we used the
modified Bayesian information criterion described by \cite{mBIC},
which has no smoothness parameter, and is implemented in the
\code{uniseg} function of the \package{cghseg}
package. 



\end{methods}


\begin{table*}[hb!]
\begin{center}
\input{table-generalization-error-global-models}
\end{center}
\caption{
  \label{table-generalization-error-global-models}
  Leave-one-out cross-validation 
  over the 6 annotated regions
  was used to estimate 
  breakpoint detection error,
  false positive (FP), and false negative (FN) rates.
  Each line shows the performance of one of the models 
  described in Section~\ref{models}.
  Models that have a smoothness parameter are shown with a colored square,
  and global and local training procedures described in 
  Section~\ref{pick} were used to 
  learn smoothness parameters.
  The global error
  is used to order the rows of the table.
  The Timings column shows the median time to fit the 
  sequence of smoothing models for a single profile.
}
\end{table*}


\section{Results}\label{results}

All the algorithms from Section~\ref{models} were applied to all the
annotated neuroblastoma copy number profiles in the data set described
in Section~\ref{data}. 
%The third major contribution of this article is
%a quantitative comparison of the performance of these common smoothing
%models on this large set of copy number profiles.  
However, the dnacopy.prune algorithm was too slow ($>24$ hours) for some
of the profiles with many data points, so these profiles were excluded
from the analysis of dnacopy.prune. Note that to decrease computation
time, the model fitting may be trivially parallelized for profiles,
algorithms, and smoothing parameter values.

\subsection{Among global models, \model{cghseg.k} exhibits the
  smallest training error in the neuroblastoma data}

The global and local training procedures were applied to the entire
set of annotated profiles. Training error curves for flsa.norm,
cghseg.k, dnacopy.sd, and glad.lambdabreak are shown in
Figure~\ref{figure-learning-curves}. Note that the global curves do
not achieve zero training error but the local curves often do,
suggesting that the local training strategy may be useful in some
cases. Also note the inflexibility of the \model{dnacopy.sd} and
\model{glad.lambdabreak} models, which do not detect a breakpoint in
profile $i=362$, even at the smallest parameter value, corresponding
to the model with the most breakpoints. Finally, note the minimum
error of $2.2\%$ achieved by \model{cghseg.k}, the global model with
the smallest training error.

The ROC curves for the training error of the global models for each
algorithm are traced in Figure~\ref{figure-roc}. It is clear that the
default parameters of each algorithm show relatively large false
positive rates. In contrast, the models chosen by maximizing agreement
with the breakpoint annotation data exhibit smaller false positive
rates at the cost of smaller true positive rates. The ROC curves
suggest that the \model{cghseg.k} algorithm is the most discriminative
for breakpoint detection in the neuroblastoma data.

\subsection{Global models generalize better than local models}
The leave-one-out cross-validation protocol was used to contrast the
test error of the models trained using the local and global 
procedures. Table~\ref{table-generalization-error-global-models} shows
the error, false positive, and false negative rates of each
model, averaged over the 6 test regions.

It is clear that the training procedure makes no difference for models
glad.default, dnacopy.default, cghseg.mBIC, and cghFLasso, which have
no smoothness parameters. The large error of these models suggest that
the assumptions of their default parameter values do not hold in the
neuroblastoma data set. More generally, these error rates suggest that
smoothness parameter tuning is critically important to obtain an
accurate smoothing of real copy number profiles.

For dnacopy.prune, glad.MinBkpWeight, glad.lambdabreak, and flsa,
there appears to be little difference between the local and global
training procedures. For models flsa.norm and cghseg.k, there seems to
be a clear advantage for the global models which share information
between profiles. Indeed, the cghseg.k model shows the minimal
estimated error of only $ 2.2\%$ on these data, with a moderately fast
median training time of 2.1 seconds.

Finally, note that the false positive rate of locally trained models
is higher than the false negative rate for most algorithms. This can be
explained by the larger fraction of normal annotations present in the
training set, and the fact that many profiles have only normal
annotations (Table~\ref{table-annotation-profile-counts}).

\subsection{Only a few profiles need to be annotated for a good global
  model}

Finally, to estimate the generalization error of a global model
trained on a relatively small training set of $t$ annotated profiles,
we applied the $n/t$-fold cross-validation procedure to the data.

For several training set sizes $t$, we plot the error of the
glad.lambdabreak, dnacopy.sd, flsa.norm, and cghseg.k models in
Figure~\ref{figure-kinetics}. It shows that adding more annotations to
the training set decreases the error in general, but at a diminishing
rate. The error curves flatten out near $t=10$, suggesting that
annotating 10 profiles is sufficient to get performance just as good
as if all the profiles were annotated. Futhermore, it is clear that
the minimum error is model-dependent, and we conjecture that it is
also annotator-dependent.

This suggests the following protocol: annotate breakpoints in about 10
profiles, then use those annotations to train a global model. In
Table~\ref{table-error-on-unseen-profiles}, we used $n/10$-fold
cross-validation to estimate the error rates of models trained using
this protocol. These error estimates are slightly larger, but the
model ordering is mostly unchanged, with respect to the leave-one-out
cross-validated estimates of the global model error rates in
Table~\ref{table-generalization-error-global-models}.  In particular,
cghseg.k still shows the best performance on these data, with an
estimated generalization error of $ 3.4\%$. 



\section{Discussion and conclusions}

We have proposed to train smoothing models using annotations
determined by visual inspection of the copy number profiles. We have
demonstrated that this approach allows quantitative comparison of
smoothing models on a new data set of 575 neuroblastoma copy number
profiles. These data provide the first set of annotations that can be
used for benchmarking the breakpoint detection ability of future
algorithms. Finally, our annotation-based approach is quite useful in
practice on real data, since it provides an objective criterion for
choosing the parameter that controls the smoothness of the model.

The clear drawback of annotation-based model selection is the time
required to create the annotations. 
However, 
% we have shown that increasing the number of annotations can
% only decrease model error, up to some limit which depends on the model
% and potentially the annotator. Also, 
we have shown diminishing returns
after about $t=10$ annotated profiles, so not much time should be
needed in general to get good performance with models trained using
annotations. Furthermore, we propose to streamline the annotation
process by using a simple, portable, free software Python annotation
GUI which is available as package \texttt{annotate\_regions} from the
Python Package Index.

% Another potential criticism of our approach is that annotations
% determined by visual inspection may not be consistent between
% experts. In
% any case, it is clear that our approach will favor algorithms that
% capture breakpoints that agree with the annotator's visual definition
% of a breakpoint.
% To get to some more objective
% criterion in the future, we could try to incoporate annotations from
% several different annotators, then combine them by estimating some
% annotator-specific false positive and false negative rates.


In contrast with our results, in a previous work on
automatic parameter tuning of smoothing models, \cite{penalized-cna}
%suggest a mathematically-motivated criterion
%based on the variance of the signal. In their work they 
claim that local models should be better in some sense: ``it is clear
that the advantages of selecting individual-specific $\lambda$ values
outweigh the benefit of selecting constant $\lambda$ values that
maximize overall performance.''  However, they do not demonstrate this
claim explicitly, and one of the contributions of this work is to show
that global models often generalize better than local models,
according to our leave-one-out estimates. 
% Intuitively, this means that
% there is the annotations of any single profile is not complete enough
% to construct an optimal profile,



We have also shown that learning a global smoothness parameter on a
limited set of annotations can generalize well to un-annotated
profiles. However, the smoothness parameterization must be carefully
chosen. For example, the flsa.norm algorithm scales the smoothness
parameter $\lambda$ by the number of points and the length of the
chromosome, and results in lower error rates than the unscaled flsa
algorithm. We are interested in investigating other parameterizations
which could reduce the error even further.



\begin{figure}[b]
  \centering
  \includegraphics[width=\columnwidth]{figure-kinetics}
  \caption{\label{figure-kinetics} Cross-validation was used to
    estimate the generalization ability of the global models with
    different sized training sets for several breakpoint detection
    algorithms. For each training set size $t$, the profiles were
    partitioned into training sets of approximately size $t$, then
    were evaluated using the annotations from all the other
    profiles. Results on these data indicate increasing accuracy (lines)
    and decreasing standard deviation (shaded bands) as the training set
    increases, with diminishing returns after approximately $t=10$,
    indicated with a vertical black line, and shown in detail for all
    algorithms in Table~\ref{table-error-on-unseen-profiles}. }
\end{figure}




We have solved the problem of smoothness parameter selection using
breakpoint annotations, but the biological question of detecting CNAs
remains. By constructing a database of annotated regions of CNAs, we
could use a similar approach to train models that detect
CNAs. Annotations could be actual copy number ($0,1,2,3,\dots$) or
some simplification (normal, deletion, amplification). For the future,
we will be interested in developing joint breakpoint detection and
copy number calling models that directly use these annotation data as
constraints or as part of the model likelihood.

It will be interesting to apply annotation-based model training to
other algorithms and data sets.  In the annotations we analyzed,
\model{cghseg.k} showed the best breakpoint detection, but another
model may be selected with another expert's annotation of the neuroblastoma
data. In future work, it will be interesting to see if our conclusions
are robust to the annotator, and generalize to data from other tumor
types. Furthermore, since next-generation sequencing data sets are
becoming more common, we plan to use visual annotations to learn
segmentation models for these data as well.


\section*{Acknowledgements}
Thanks to Edouard Pauwels for helpful comments on an early draft of
the paper.
 



\begin{table}[b]
\begin{center}
\input{table-error-on-unseen-profiles}
\end{center}
\caption{\label{table-error-on-unseen-profiles}
  The $n/t$-fold cross-validation protocol was used to estimate
  error, false positive (FP), and false negative (FN) rates.
  Mean and standard deviation (sd) over
  $\lfloor n/t\rfloor =57$ folds are shown as percents.
  Squares show the same colors as in the figures, 
  and are absent for models that have no smoothness parameters.
  The smoothness parameter was
  chosen using annotations from approximately $t=10$ profiles. 
}
\end{table}




\paragraph{Funding\textcolon}
% • The sentence should begin: ‘This work was supported by …’ • The
% full official funding agency name should be given, i.e. ‘National
% Institutes of Health’, not ‘NIH’ (full RIN-approved list of UK
% funding agencies) Grant numbers should be given in brackets as
% follows: ‘[grant number xxxx]’ • Multiple grant numbers should be
% separated by a comma as follows: ‘[grant numbers xxxx, yyyy]’ •
% Agencies should be separated by a semi-colon (plus ‘and’ before the
% last funding agency) • Where individuals need to be specified for
% certain sources of funding the following text should be added after
% the relevant agency or grant number 'to [author initials]'.

% An example is given here: ‘This work was supported by the National
% Institutes of Health [AA123456 to C.S., BB765432 to M.H.]; and the
% Alcohol & Education Research Council [hfygr667789].’

This work was supported by Digiteo [DIGITEO-BIOVIZ-2009-25D to
T.D.H.]; the European Research Council [SIERRA-ERC-239993 to F.B; SMAC-ERC-280032 to J-P.V.];
the French National Research Agency [ANR-09-BLAN-0051-04 to J-P.V.]; the Annenberg Foundation [to G.S.];
the French Programme Hospitalier de Recherche Clinique [PHRC IC2007-09
to G.S.]; the French National Cancer Institute [INCA-2007-1-RT-4-IC to
G.S.]; and the French Anti-Cancer League.

%TDH was supported by
%grant DIGITEO-BIOVIZ-2009-25D.  
 
%FB was supported by grant SIERRA-ERC-239993. 
%JPV was supported by grants ANR-07-BLAN-0311-03 and
%ANR-09-BLAN-0051-04.
%
%GS was supported by the Annenberg Foundation, PHRC IC2007-09 and INCA
%2007-1-RT-4-IC
%
%GS is supported  by the Annenberg Foundation
%
%Isabelle: U830 Inserm laboratory was supported by grants from the
%Ligue Nationale contre le Cancer (Equipe labellisée).
%
%Associations : the Ligue Nationale contre le Cancer (Equipe
%labellisée), the Association Hubert Gouin, Les Bagouz à Manon, les
%amis de Claire and Enfance et Santé
%



\paragraph{Conflict of interest\textcolon} None declared.

\newpage
 \bibliographystyle{natbib}

\bibliography{refs}


\end{document}
